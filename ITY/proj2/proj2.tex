\documentclass[a4paper, 11pt]{article}
\usepackage[left=1.5cm,text={18cm, 25cm},top=2.5cm]{geometry}
\usepackage[utf8]{inputenc}
\usepackage[IL2]{fontenc}
\usepackage[czech]{babel}
\usepackage{times}
\usepackage{geometry}
\usepackage{cuted}
\usepackage{lipsum}
\usepackage{amsmath}
\usepackage{amsthm}
\usepackage{amsfonts}
\newcommand{\myuv}[1]{\quotedblbase #1\textquotedblleft}
\newcommand{\mycode}[1]{\texttt{#1}}
\theoremstyle{definition}
\newtheorem{definice}{Definice}
\newtheorem{veta}{Věta}
\setlength{\emergencystretch}{3em}

\begin{document}
\begin{titlepage}
\begin{center}
\Huge
\textsc{Fakulta informačních technologií \\ Vysoké učení technické v Brně}\\
\vspace{\stretch{0.382}}
Typografie a publikování -- 2. projekt\\ Sazba dokumentů a matematických výrazů\\
\vspace{\stretch{0.618}}
\end{center}
{\LARGE 2018 \hfill Kateřina Fořtová (xforto00)}
\end{titlepage}

\begin{twocolumn}
\section*{Úvod}
V~této úloze si vyzkoušíme sazbu titulní strany, matematických
vzorců, prostředí a dalších textových struktur obvyklých
pro~technicky zaměřené texty (například rovnice (1)
nebo Definice 1 na strane 1). Rovnež si vyzkoušíme používání
odkazu \verb|\ref| a \verb|\pageref|.
Na~titulní straně je využito sázení nadpisu podle optického
středu s~využitím zlatého řezu. Tento postup byl
probírán na~přednášce. Dále je použito odřádkování se~zadanou relativní velikostí 0.4em a 0.3em.
\section{Matematický text}
Nejprve se podíváme na~sázení matematických symbolů
a výrazů v~plynulém textu včetně sazby definic a vět s~využitím
balíku \verb|amsthm|. Rovnež použijeme poznámku podčarou s~použitím příkazu \verb|\footnote|. Někdy je vhodné
použít konstrukci \verb|${}$|, která říká, že matematický text
nemá být zalomen.

\definice Turingův stroj \textit{(TS) je definován jako šestice tvaru} $M = (Q, \Sigma, \Gamma, \delta, q_0, q_F)$\textit{, kde:}
\begin{itemize}
\item{$Q$\textit{ je konečná množina} vnitřních (řídících) stavů,}
\item{$\Sigma$\textit{ je konečná množina symbolů nazývaná }\textmd{vstupní abeceda, }$\Delta\notin\Sigma$}
\item{$\Gamma$\textit{ je konečná množina symbolů, }$\Sigma\subset\Gamma$, $\Delta\in\Gamma$\textit{, nazývaná }\textmd{pásková abeceda,}}
\item{$\delta$: ($Q$$\setminus$$\{$$q_{F}$$\}$)$\times$$\Gamma$$\rightarrow$$Q$$\times$($\Gamma$$\cup$$\{$L$, $R$\}$)\textit{, kde }$L$, $R$ $\notin$ $\Gamma$\textit{, je parciální} přechodová funkce,}
\item{$q_0$ \textit{je} počáteční stav, $q_0$ $\in$ $Q$ $a$}
\item{$q_F$ \textit{je} koncový stav, $q_F$ $\in$ $Q$}
\end{itemize}

Symbol $\Delta$ značí tzv. \textit{blank} (prázdný symbol), který
se vyskytuje na~místech pásky, která nebyla ještě použita
(může ale být na~pásku zapsán i později).

\textit{Konfigurace pásky} se skládá z~nekonečného řetězce, který reprezentuje obsah pásky a pozice hlavy na~tomto řetězci. Jedná se o~prvek množiny $\{$$\gamma$$\Delta^\omega$$\mid$$\gamma$$\in$$\Gamma^\ast$$\}$$\times$$\mathbb{N}$.\footnote{Pro~libovolnou abecedu $\Sigma$ je $\Sigma^\omega$ množina všech nekonečných řetězců nad $\Sigma$, tj. nekonečných posloupností symbolů ze $\Sigma$. Pro připomenutí: $\Sigma^\ast$ je množina všech konečných řetězců nad $\Sigma$.}

\textit{Konfiguraci pásky} obvykle zapisujeme jako\newline $\Delta$$x$$y$$z$$\underline{z}$$x$$\Delta$\,\dots(podtržení značí pozici hlavy). \textit{Konfigurace stroje} je pak dána stavem řízení a konfigurací pásky. Formálně se jedná o prvek množiny\newline$Q$ $\times$ $\{$$\gamma$$\Delta^\omega$$\mid$$\gamma$$\in$$\Gamma^\ast$$\}$$\times$$\mathbb{N}$.

\subsection{Podsekce obsahující větu a odkaz}
\definice Řetězec $w$ nad abecedou $\Sigma$ je přijat TS \textit{M\\ jestliže M při aktivaci z počáteční konfigurace pásky}\\ $\underline{\Delta}$$w$$\Delta$\,\dots \textit{a počátečního stavu} $q_0$ \textit{zastaví přechodem\\do koncového stavu} $q_F$\textit{, tj.} ($q_0$, $\Delta$$w$$\Delta^\omega$, 0)$\underset{M}{\overset{\ast}{\vdash}} (q_F, \gamma, n)$\\\textit{pro nějaké} $\gamma$$\in$$\Gamma^\ast$ \textit{a} $n$ $\in$ $\mathbb{N}$.

\textit{Množinu} $L(M)$ = $\{$$w$ $\mid$ $w$ \textit{je přijat} $T$$S$ $M$$\}$ $\subseteq$ $\Sigma^\ast$\\\textit{nazýváme} jazyk přijímaný TS $M$.

Nyní si vyzkoušíme sazbu vět a důkazů opět s použitím balíku \verb|amsthm|.
\veta Třída jazyků, které jsou přijímány TS, odpovídá rekurzivně vyčíslitelným jazykům.

\begin{proof}
V důkaze vyjdeme z Definice 1 a 2.
\end{proof}

\section{Rovnice a odkazy}

Složitější matematické formulace sázíme mimo plynulý text. Lze umístit několik výrazů na~jeden řádek, ale pak je třeba tyto vhodně oddělit, například příkazem \verb|\quad|.

\begin{eqnarray*}
\sqrt[i]{x_i^3} & \textrm{kde $x_i$ je $i$-té sudé číslo} & y_i^{2\cdot {y_i}} \neq y_i^{{y_i}^y_{i}}
\end{eqnarray*}

V rovnici (1) jsou využity tři typy závorek s~různou explicitně definovanou velikostí.

\begin{eqnarray}
x & = & \Bigg\{ \bigg( \Big[a + b \Big] \ast c \bigg )^d \oplus 1 \Bigg\}\\
y & = & \lim_{x \to \infty}\frac{sin^{2}x + cos^{2}x}{\frac{1}{\log_{10}x}} \nonumber
\end{eqnarray}

V této větě vidíme, jak vypadá implicitní  vysázení limity $\lim_{n \to \infty}f(n)$ v normálním odstavci textu. Podobně je to i s~dalšími symboly jako $\sum\limits _{i=1}^n 2^i$ či $\bigcup_{A\in B}A$. V případě vzorců $\lim_{n \to \infty}f(n)$ a $\sum\limits _{i=1}^n 2^i$ jsme si vynutili méně úspornou sazbu příkazem \verb|\limits|.

\begin{eqnarray}
\int_a^b \mathrm{f(x)}\,\mathrm{d}x & = & -\int_b^a \mathrm{g(x)}\,\mathrm{d}x
\end{eqnarray}

\begin{eqnarray}
\overline{\overline{A \vee B}} & \Leftrightarrow & \overline{\overline{A} \wedge \overline {B}}
\end{eqnarray}

\section{Matice}
Pro sázení matic se velmi často používá prostředí \verb|array| a závorky (\verb|\left|, \verb|\right|).

$
\begin{pmatrix} 
a + b & \widehat{\xi + \omega} & \widehat{\pi} \\
\overrightarrow{a} & \overleftrightarrow{AC} & \beta
\end{pmatrix}
$
$
= 1
$
$
\iff
$
$
\mathbb{Q}
$
$
=
$
$
\mathbb{R}
$
$ A = $
$
\begin{Vmatrix}
  a_{11} & a_{12} & \cdots & a_{1n} \\
  a_{21} & a_{22} & \cdots & a_{2n} \\
  \vdots  & \vdots  & \ddots & \vdots  \\
  a_{m1} & a_{m2} & \cdots & a_{mn}
\end{Vmatrix}
$
$ = $
$
\begin{vmatrix} 
t & u \\
v & w \\
\end{vmatrix}
$
$ = tw - uv $

Prostředí \verb|array| lze úspěšně využít i jinde.

$$
\binom{n}{k} = \left\{ 
\begin{array}{ll} 
\frac{n!}{k!(n-k)!}&\text{pro $0 \le k \le n$} \\ 
0&\text{pro $k < 0$ nebo $k > n$}\\  
\end{array} \right. 
$$

\section{Závěrem}
V~případě, že budete potřebovat vyjádřit matematickou konstrukci nebo symbol a nebude se Vám dařit jej nalézt v~samotném \LaTeX{u}, doporučuji prostudovat možnosti balíku maker \AmS--\LaTeX.

\end{twocolumn}

\end{document}
